\documentclass{article}
\usepackage[utf8]{inputenc}

\title{EDA031 - News Project}
\author{Emil Westenius, Erik Westenius, Niklas Ovnell \& Johan Holm (email) }
\date{April 2015}

\begin{document}

\maketitle
\pagebreak
\section{Introduction}
The objective of this project is to implement a news server and news client. The server contains a database with newsgroups and articles. A user connects to the server using the client. The client has a text-based user interface which allows users to create and delete news groups and articles. 

\section{tillfälligt}
OBS\hfill\\
"Create a directory username (i.e., your own username) with subdirectories src and bin.
Collect all source files in the src directory and write a makefile. make all should create the
executables (the two versions of the server and the client), make install should copy the
executables to the bin directory."\hfill\\
Kul sak jag nyss hittade i sharelatex: ctrl + alt + piltangenter för att sätta markörer på flera rader samtidigt (op!). Esc för att avbryta.

Om någon lägger till grejer i mappen utifrån klienten kaosar allt

\section{Requirements}
% lägga i conclusion istället?
\begin{itemize}
\item Each newsgroup has a unique name and a unique ID greater than zero.
\item Each article has a title, an author, a text and a unique ID greater than zero.
\item Newsgroup and article IDs cannot be reused. 
\end{itemize}

\section{System Design}
% UML-figurer

\section{Classes}
\subsection{\texttt{Server}}
The server listens to a port and handles multiple connections.

\subsection{\texttt{Connection}}
Used to connect the server with the client. 

\subsection{\texttt{MessageHandler}}
Uses \texttt{protocol.h} to decode and encode messages.

\subsection{\texttt{MyServer}}
Main program for the server. Run by the user to start a server with a given host (hur vara formulera sig?) and port.

\subsection{\texttt{MyClient}}
Main program for the client. Run by the user to start a client that connects to a given port. 

\subsection{\texttt{Database}}
Database interface. Contains member functions for creation, deletion and listing of news groups and their articles. 

\subsection{\texttt{InMemoryDatabase}}
This database handles everything in the primary memory and is initiated from scratch every time the user starts a server.

\subsection{\texttt{FileSystemDatabase}}
This database uses a file system to store news groups and articles between runs. Newsgroups are represented by folders named after newsgroup ID. Each folder contains a meta file with the folder name and the ID that the next created article will get. A newsgroup folder also contains one file for each article. These files are named after the article IDs and contain title, author and article text. 

\subsection{\texttt{Newsgroup}}
A newsgroup is identified by a unique name and ID number. It may contain any number of articles.

\subsection{\texttt{Article}}
An article is identified by a unique ID number. Each article has a title and an author and contains a text with the contents of the article. % fult?

\section{Conclusion}


\end{document}
